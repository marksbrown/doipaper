\subsection{Determining the Time Resolution}
There are three principal sources of systematic error we wish to correct for, before attempting to determine the time resolution of a given scintillator detector. Firstly low-energy ($\lesssim$1MeV) gamma ray photons may interact with matter in two ways. We wish to select events which interact solely by the photoelectric effect to ensure there is no additional time ambiguity introduced by scattering within the system. To do this we fit to the `photopeak'\footnote{As the energy resolution is also not perfect, the 0.511keV gamma ray photon will be absorbed generating a Gaussian distribution known as the photopeak in the energy specta of the scintillator detector} generated by the total absorption of the gamma ray photon within the scintillator detector and exclude events outside a $2\sigma$ window about the peak location. Secondly, the number of `edges' recorded by the oscilloscope per sampling period will vary due to overlapping independent interactions within the scintillator detector. As a given logical pulse will generate two such edges, an ambiguity in the arrival time and the energy of a given interaction if multiple events occur within the same sampling period. The time information of a given signal is encoded in the first edge of the logical pulse. The width of the pulse itself is related to the energy deposited within the scintillator detector. Thus by selecting for sampling periods containing only two edges we remove this error. Finally we reject events where the difference in arrival time between the two scintillator detectors is greater than a prechosen limit. In this way, we reduce the chance of random events triggering the scintillator detectors simultaneously. This limit is typically no greater than a nanosecond.

\textrm{CTR} = 4\sqrt{\ln{2}}\sqrt{\sigma-\sigma_\textrm{ref}^2}
\label{eqn:ctr-fromreference}
\end{align}


