\section{Introduction} 
Determination of the depth of interaction (DOI) is of importance for PET to reduce the cross-section of each scintillator crystal without degrading the overall sensitivity\cite{Moses_2001}\cite{Humm_Rosenfeld_Del_Guerra_2003}. In doing so parallax error and off-axis spatial resolution can be improved. Within monolithic scintillator detectors the same DOI information allows spatial confinement within the detector \cite{am_Borghi_Seifert_Schaart_2013}\cite{Maas_Bruyndonckx_Schaart_2012}; thus potentially allowing more novel\cite{Dendooven_Lohner_Beekman_2009}\cite{n_der_Lei_van_Dam_Schaart_2013} layouts and geometries.

Two geometries, two scintillator crystal materials and three wrapping configurations are explored in this work. Namely two identical $2\times2\times30$mm$^3$ Proteus LYSO:Ce and a single $2\times2\times20$mm$^3$ Ca-co-doped LSO:Ce. Furthermore a $2\times2\times5$mm$^3$ Ca-co-doped LSO:Ce is used as the reference scintillator. All scintillator crystals are polished. PTFE (Teflon) tape is used as the wrapping material due it's ubiquity in PET and it's diffusive properties.

In this paper we explore the relationship between the interaction position position of 0.511MeV gamma ray photons and the performance of the scintillator detector. The shortest distance from the interaction position to the photodetector, referred to as the depth of interaction (DOI), is a potential source of degradation to the timing and energy performance of the scintillator detector.
