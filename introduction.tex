\section{Introduction} 
Determination of the depth of interaction (DOI) is of importance for PET to reduce the cross-section of each scintillator crystal without degrading the overall sensitivity\cite{Moses_2001}\cite{Humm_Rosenfeld_Del_Guerra_2003}. Within monolithic scintillator detectors the same property allows spatial confinement \cite{am_Borghi_Seifert_Schaart_2013}\cite{Maas_Bruyndonckx_Schaart_2012}.

The standard timing coincidence apparatus is altered in two key ways. Firstly the scintillator detector under interest is arranged perpendicular to the reference detector. Secondly the gamma ray source is positioned near to the scintillator detector. This arrangement is shown schematically in figure \ref{fig:doi-ctr}. Electronic collimation due to $\gamma\gamma$ correlation defined a DOI region within the scintillator crystal under investigation. By moving the scintillator detector vertically, with respect to the reference detector and source, the DOI region will move. The size of the DOI region is determined by the separation distances between scintillator detectors and the source. In this work the Na22 source has a diameter of 1mm. Where the source is much closer to the scintillator detector under interest than to the reference detector, the DOI region will tend to the width of the source.

In this paper we explore the relationship between DOI and the timing performance due to scintillator crystal geometry, material and wrapping. Three scintillator crystals are investigated; Namely two identical $2\times2\times30$mm$^3$ Proteus LYSO:Ce and a $2\times2\times20$mm$^3$ Ca-co-doped LSO:Ce. All scintillator crystals are polished. PTFE (Teflon) tape is used as the wrapping material.
