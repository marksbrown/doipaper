\section{Introduction} 
Determination of the depth of interaction (DOI) is of importance for PET to reduce the cross-section of each scintillator crystal without degrading the overall sensitivity\cite{Moses_2001}\cite{Humm_Rosenfeld_Del_Guerra_2003}. In doing so parallax error and off-axis spatial resolution can be improved. Within monolithic scintillator detectors the same DOI information allows spatial confinement within the detector \cite{am_Borghi_Seifert_Schaart_2013}\cite{Maas_Bruyndonckx_Schaart_2012}; thus potentially allowing more novel\cite{Dendooven_Lohner_Beekman_2009}\cite{n_der_Lei_van_Dam_Schaart_2013} layouts and geometries.

In this paper we explore the relationship between DOI and the timing performance due to scintillator crystal geometry, material and wrapping. Three scintillator crystals are investigated; Namely two identical $2\times2\times30$mm$^3$ Proteus LYSO:Ce and a $2\times2\times20$mm$^3$ Ca-co-doped LSO:Ce. All scintillator crystals are polished. PTFE (Teflon) tape is used as the wrapping material.
