\section{Introduction} 
Detection of ionising radiation is typically accomplished by transducing the incoming particle into light. This light can then be converted to an electrical signal and subsequently analysed. Scintillator detectors are comprised of three primary components, as shown in figure \ref{fig:doi-ctr}. Namely a scintillator crystal for creation of thousands of optical photons, a photodetector for conversion of the light to an electrical signal and a layer of optical grease between the two components to improve coupling. Reductions in the scintillator detector detection time uncertainty, known as the time resolution, are important for reducing statistical noise in positron emission tomography (PET) images\cite{Moses_Ullisch_2006}.

In this work we investigate the relationship between the interaction position of 0.511 MeV gamma ray photons and the timing and energy performance of the scintillator detector. The depth of interaction (DOI), shown in figure \ref{fig:doi-ctr}, is the shortest distance to the photodetector from the gamma ray photon ($\gamma$) interaction position. The DOI is a potential source of degradation to the timing and energy performance of the scintillator detector due to photon time of flight and light loss from increased path lengths within the scintillator crystal. Furthermore determination of the DOI, of a given interaction, is of importance for PET to negate or reduce the contribution of parallax error upon the spatial resolution \cite{Moses_2001}\cite{Humm_Rosenfeld_Del_Guerra_2003}. If successful, longer scintillator crystals may be used leading to an improvement in the PET scanner's sensitivity and reduce overall scan times. Within monolithic scintillator detectors the same DOI information allows spatial confinement within the detector itself \cite{am_Borghi_Seifert_Schaart_2013}\cite{Maas_Bruyndonckx_Schaart_2012}, thus potentially allowing more novel\cite{Dendooven_Lohner_Beekman_2009}\cite{n_der_Lei_van_Dam_Schaart_2013} layouts and geometries.

In this paper we begin by describing the standard and DOI coincidence apparatus, along with the method utilised in both for analysing the raw data in section \ref{sec:method}. Using this method we characterise the $2\times2\times5$mm$^3$ Agile Ca-co-doped LSO:Ce scintillator crystal used in the reference scintillator detector in section \ref{sec:reference}. Once this accomplished the time resolution with scintillator crystal length ($L$) is explored with the standard coincidence apparatus using two identical $2\times2\times L$mm$^3$ Proteus LYSO:Ce scintillator crystals in section \ref{sec:standardctr}. Measurements conducted using the DOI coincidence apparatus are split into two. Firstly for two identical $2\times2\times30$mm$^3$ Proteus LYSO:Ce and secondly for a single $2\times2\times20$mm$^3$ Agile Ca-co-doped LSO:Ce. These are covered in sections \ref{sec:30mm} and \ref{sec:20mm}. In doing so we explore the contribution of scintillator crystal material, geometry and wrapping. All scintillator crystals are polished. PTFE (Teflon) tape is used as the wrapping material due its diffusive properties. Finally, we discuss the results in the discussion in section \ref{sec:discussion}.
