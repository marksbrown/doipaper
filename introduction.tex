\section{Introduction} 
The depth of interaction (DOI), shown in figure \ref{fig:doi-ctr}, is the shortest distance to the photodetector from the interaction position. Determination of the DOI is of importance for positron emission tomography (PET) to negate or reduce the contribution of parallax error upon the spatial resolution \cite{Moses_2001}\cite{Humm_Rosenfeld_Del_Guerra_2003}. Longer scintillator crystals may be used without increased spatial resolution degradation to improve the PET scanner's overall sensitivity and reduce scan time. Within monolithic scintillator detectors the same DOI information allows spatial confinement within the detector \cite{am_Borghi_Seifert_Schaart_2013}\cite{Maas_Bruyndonckx_Schaart_2012}, thus potentially allowing more novel\cite{Dendooven_Lohner_Beekman_2009}\cite{n_der_Lei_van_Dam_Schaart_2013} layouts and geometries.

We explore the relationship between the DOI of 511keV gamma ray photons and the timing and energy performance of the scintillator detector. The DOI is a potential source of degradation to the timing and energy performance of the scintillator detector due to photon time of flight and light loss from increased path lengths within the scintillator crystal.

We begin by outlining the method utilised in the standard and DOI coincidence measurements in section \ref{sec:method}. Following this the reference detector (a $2\times2\times5$mm$^3$ Ca-co-doped LSO:Ce) is characterised as well as varying scintillator crystal lengths in sections \ref{sec:reference} and \ref{sec:standardctr} respectively. Next we look at the DOI results for two identical $2\times2\times30$mm$^3$ Proteus LYSO:Ce followed by a single $2\times2\times20$mm$^3$ Agile Ca-co-doped LSO:Ce in sections \ref{sec:30mm} and \ref{sec:20mm}. This is to determine the contribution of scintillator crystal material, geometry and wrapping. All scintillator crystals are polished. PTFE (Teflon) tape is used as the wrapping material due its diffusive properties. Finally, we bring the results together in the discussion in section \ref{sec:discussion}.
