\subsubsection{Other LSO:CeCa 20mm Results}
In table \ref{tab:ctrfit-20-results} we again perform the two polynomial fits. In this case both the fit to the intercept and to a linear relationship perform well. Again the standard deviation is on the order of the error in the intercept$_\text{notfit}$. We conclude no relationship between the CTR and the DOI is evident, within error. 

In figure \ref{fig:centroidposition-20} the same plateau as observed in the 30mm measurements is again seen. In this case the plateau is reached close to 10mm for the wrapped measurements. Again the peak to peak range in the delay peak centroid is comparable to the scintillator crystal length. We can conclude that the shift is predominantly dependent upon the geometry of the scintillator crystal. 

In figure \ref{fig:lightoutput-20} we see the right photopeak centroid with DOI. In this case the plateau commences at (or near to) 10mm. As this has changed with scintillator crystal length the geometry is clearly the primary contribution due to this effect. The peak to peak ranges are different between configurations. Namely 120$\pm$12ps, 162.2$\pm$1.7ps and 130.7$\pm$1.5ps for the unwrapped, partially wrapped and wrapped configurations. In the 30mm case these values are 227.9$\pm$2.5ps and 223.3$\pm$2.5ps for the unwrapped and wrapped configurations respectively. We conclude two things:

\begin{enumerate}
\item Larger peak to peak range on the partially wrapped case is due to...
\item The unwrapped and wrapped configurations have the same peak to peak range because...
\end{enumerate}