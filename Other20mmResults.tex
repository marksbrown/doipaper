\subsubsection{Other LSO:CeCa 20mm Results}
In table \ref{tab:ctrfit-20-results} we again perform the two fittings. In this case both the fit to the intercept and to a linear relationship perform well. We also see that the peak-to-peak change is on the other of the error in the intercept values. From these second set of measurements we would conclude no relationship between CTR and DOI within error. This is despite these measurements resulting in a a lower error per CTR measurement and a greater number of $\gamma\gamma$ events recorded; namely 5000 compared to 3400.

In figure \ref{fig:centroidposition-20} the same plateau as observed in the 30mm measurements is again seen. In this case the plateau is reached close to 10mm for the wrapped measurements. Again the peak to peak range in the delay peak centroid is comparable to the scintillator crystal length. We can conclude that the shift is predominantly dependent upon the geometry of the scintillator crystal. 

In figure \ref{fig:energyresolution-20} we see the energy resolution improve with increasing DOI. In figure \ref{fig:lightoutput-20} we see the right photopeak centroid with DOI. We see that the energy resolution is poorest at low DOI; this despite the light output being higher at low DOI. The improvement in the energy resolution from unwrapped to wrapped is as expected corresponding to the increase in the light output. As the increase in the light output is large when covering the face opposing the photodetector, it is clearly important that in real usage it is ensured that a proper PTFE wrap is in place to maximise timing and energy performance.