\subsection{Standard Coincidence Measurements}
In the standard coincidence apparatus, the two scintillator detectors are placed opposite one another with the source an equal distance between the two.

\subsection{Depth Of Interaction Coincidence Measurements}
The apparatus as described in \cite{arron_Meyer_Pauwels_Lecoq_2012} is altered in several respects as shown in \ref{fig:actualsetup}. Firstly the right photodetector is placed within a 3D-printed clamp, shown in figure \ref{fig:actualsetup}, designed to hold the scintillator crystal vertically with respect to the reference detector. Secondly the Na22 source is moved to a 5mm separation distance from the scintillator crystal under investigation. As in the standard apparatus both scintillator crystals are coupled to the Hamamatsu MPPC S10931-050P SiPM photodetectors using Rhorosil 47A optical grease. 

\subsection{Reference Coincidence Measurements}
The reference scintillator crystal, shown on the left of \ref{fig:actualsetup}, is a 2×2×5mm3 Ca-co-doped LSO:Ce wrapping in PTFE tape coupled. Using two identical such crystals the coincidence time resolution was determined using the standard coincidence apparatus, values shown in table \ref{tab:referencevals}. This value is agreement with prior measurement\cite{arron_Meyer_Pauwels_Lecoq_2012}.