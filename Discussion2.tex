\section{Discussion}
The most notable previous measurement of the CTR with DOI was conducted by Moses\cite{Moses1999}. In this paper the characteristic shift in the delay peak position with DOI was attributed to the time difference between the forward and backward modes approaching negligible levels thus the electronics will `see' a narrower, higher density pulse of photons compared to two individual pulses leading to an averaging effect related to the geometry of the scintillator crystal. 

In this chapter the results shown in sections \ref{sec:unwrappedthirty} and \ref{sec:extendedduration} show the lowest error per measurement, whilst demonstrating the clearest null relationship between the DOI and CTR.  Of interest is the drop in the mean CTR from $207\pm1$ to $199\pm1$ with the inclusion of PTFE tape wrapping. Whilst a drop is expected due to the increased light output with wrapping, this drop is not as high as might expect due to the $\frac{1}{\sqrt{N}}$ dependency of the CTR. This implies only a small portion of the generated photons by the scintillator crystal are `useful' for timing purposes.

The results presented will subsequently be published, however the results are in direct contradiction with \cite{Yeom2013}. In this the CTR is shown to degrade with reduced DOI with range of 160 to 190 ps over a $3\times3\times20$\mmc LYSO scintillator crystal. In \cite{Bircher2012} however no relationship between CTR and DOI is visible, on the provision that the scintillator crystal surface is smooth.

The second result the DOI results show, is a consistent shift the right photopeak position with DOI. This drop, typically between 4\% and 10\% over the length of the scintillator crystal, indicates a loss mechanism with increasing DOI. A decrease in the light output with scintillator crystal length\cite{Moszyixki1997} is potentially the same effect. Also the mean photon path length\footnote{Investigate this via Monte Carlo} increases with DOI the chance of absorption within the scintillator crystal is increased. Another potential source of loss is due to imperfections in the surface of the scintillator crystal. For a small perturbation from specular, a situation that can be described adequately by the Lobe reflection model\cite{Janecek2010a}, a small portion of light can couple out of the scintillator.