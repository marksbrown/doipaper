\section{Experimental Method}
\begin{itemize}
 \item Scintillator crystals stored out of direct sunlight as well as strong 
lighting. Ideally crystals should be processed under a red light alone.
 \item Scintillator crystals should be cleaned with Ethanol using carbon-tipped 
tweezers on lens tissue to prevent scratching the surfaces.
 \item To wrap the scintillator crystal, first cut a long length of PTFE 
(Teflon) tape. For 30mm crystal a length of approximately half a metre is 
needed. If not wrapped skip to adding grease.
\begin{itemize}
 \item Begin by pressing the scintillator crystal onto one edge of the tape 
with the tweezers. In doing so, when we begin rolling the tape will adhere to 
the crystal. Leave approximately half a millimetre unwrapped at one end to 
ensure one face is not covered with tape. 
 \item Begin rolling the tape about the crystal. Ensure a tight wrap by pulling 
tight with the tweezers. Do NOT use bare hands. Ensure at least 5 layers over 
all faces of the scintillator crystal.
 \item Twist tape at end of crystal tightly to ensure this small face is 
properly covered.
 \item Press tape into crystal using tweezers to ensure no air remains inside 
the tape. Tape should add at least 0.5mm to the thickness of a scintillator 
crystal.
\end{itemize}
\item Optical grease with a sufficiently high viscosity can be applied to 
the unwrapped end of the scintillator crystal by dip or dropping using a small 
cocktail stick. Otherwise, grease should be applied directly to the 
photodetector.
\end{itemize}