In figure \ref{fig:ctrdoi-20} we see the CTR improves (as in decreases) with increasing amounts of wrapping by 20ps; from unwrapped, to partially wrapped and finally to wrapped. This is due to increased light output as seen in figure \ref{fig:lightoutput-20}. These measurements have lower individual CTR errors than the previous 30mm measurements, likely due to a greater number of $\gamma\gamma$ events recorded. These being namely 5000 for LSO:CeCa compared to 3400 for LYSO:Ce. As the volume of the confinement region within the scintillator crystal is the same in both measurements this difference is primarily due to the difference in measurement time. The wrapped CTR is $198\pm2$ps compared to 176$\pm$7ps for the standard coincidence measurement given in [Table 2]\cite{uffray_Jarron_Meyer_Lecoq_2014}. For an equivalent $2\times2\times20$mm$^3$ LYSO:Ce scintillator crystal a CTR of 202.7$\pm$4.0ps in standard coincidence is observed. Therefore we see that LSO:CeCa is a superior material to that of LYSO:Ce. We also see that the CTR is worse in the DOI timing coincidence apparatus than in the standard. This conclusion is in agreement with the 30mm measurements where we see values of 209.6$\pm$3.8ps for the standard coincidence to 239.1$\pm$1.8ps in the DOI coincidence apparatus.