\subsection{Contribution of the Threshold Voltage}
\begin{table}
\caption{\label{tab:thresholdtest} Key values from threshold voltage test}
\begin{tabular}{llllr}
\hline
Threshold (mV) & PTP change (ps) & loc &Delay Peak Centroid (ps) & CTR (ps) & $\chi^2_\text{Reduced}$\\
\hline
80        &  103$\pm$35 &  346.7$\pm$6.9 &  308.3$\pm$23.4 &        0.61 \\
200       &   36$\pm$17 &  550.2$\pm$3.6 &  337.4$\pm$12.2 &        0.95 \\
600       &   35$\pm$15 &  789.8$\pm$3.6 &  358.7$\pm$11.9 &        1.01 \\
1000      &   38$\pm$16 &  898.6$\pm$3.7 &  359.5$\pm$11.8 &        1.06 \\
\hline
\end{tabular}
\end{table}

A final test was performed on the wrapped $2\times2\times20$mm$^3$ LSO:CeCa scintillator crystals; Namely that the threshold of the discriminator coupled to the scintillator scintillator was increased from it's value of 80mV, to 200mV, 600mV and 1000mV. In figure \ref{fig:thresholdtest} we present the delay peak centroid for each threshold with DOI as a ratio of the lowest DOI value in the measurement. Thus we can see how the gradient varies between identical apparatus due to threshold increment alone. It can be seen that the gradient is shallower with increasing threshold. Secondly we can see that the peak to peak range is lower with increasing voltage. Both results indicate that the higher photon density required to trigger the photodetector at higher threshold voltages is related to the geometry of the scintillator crystal.

