\subsection{Contribution of the Threshold Voltage}

A final test was performed on the wrapped $2\times2\times20$mm$^3$ LSO:CeCa scintillator crystals; Namely that the threshold of the discriminator coupled to the scintillator under investigation was increased from its optimal value of 80mV to 200mV, 600mV and 1000mV. In figure \ref{fig:thresholdtest} we present the delay peak centroid for each threshold with DOI as a ratio of the lowest DOI value in the measurement. Thus we can see how the gradient varies between identical apparatus due to threshold increment alone. It can be seen that the gradient is shallower with increasing threshold. Describing the gradient as an effective refractive index we see


80mV for < 12mm n=3.5+/-1.2
200mV for < 12mm n=3.7+/-1.6
600mV for < 12mm n=3.9+/-2.2
1000mV for < 12mm n=3.9+/-2.1


Secondly, in table \ref{tab:thresholdtest}, we can see that the peak to peak range is significantly lower at higher voltages. Both results indicate that the higher photon density required to trigger the photodetector at higher threshold voltages is related to the geometry of the scintillator crystal.

