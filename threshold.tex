\subsection{Contribution of the Threshold Voltage}
\label{sec:threshold}
A final test was performed on the wrapped $2\times2\times20$mm$^3$ LSO:CeCa scintillator crystals; Namely that the threshold of the discriminator coupled to the scintillator under investigation was increased from its optimal value of 80mV to 200mV, 600mV and 1000mV. The same (lack of) relationship between the CTR and DOI was seen regardless of threshold voltage, indicating that the discriminator was not a factor. In figure \ref{fig:thresholdtest} we present the delay peak centroid for each threshold with DOI as a ratio of the lowest DOI value in the measurement. With increasing threshold voltage we see that the right photodetector triggers later and thus causes the photopeak centroid to shift further to the right. In table \ref{tab:thresholdtest}, the mean values across all DOI are shown. We can see that the peak to peak range, defined as the maximum change in the CTR over the DOI, is significantly lower at higher voltages. Both results indicate that the higher photon density required to trigger the photodetector at higher threshold voltages is related to the geometry of the scintillator crystal.

