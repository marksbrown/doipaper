\subsection{20mm LSO:CeCa Measurements}
To determine to what degree the material and scintillator crystal length contributes to the timing and energy performance we repeat the above measurements with a polished $2\times2\times20$mm$^3$ Ca-co-doped LSO:Ce scintillator crystal. In addition we consider a third configuration; namely that we wrap the side faces but leave the face opposing the photodetector unwrapped. In doing so we will reduce the light output and thus the contribution from the backwards reflecting mode. In these sets of measurements we move the reference detector 10mm further from the Na22 source and collect each measurement for 90 minutes.

In figure \ref{fig:confinement-20} the confinement is consistent across DOI except for the unwrapped configuration. In this case we see a gradual drop off in number of $\gamma\gamma$ events recorded with increasing DOI. This is due to poor vertical alignment such that the DOI region lies to the side of the scintillator crystal at higher DOI not within it. In figure \ref{fig:ctrdoi-20} we see that this results increasingly larger errors in the CTR until not enough values are collected to accurately determine the value at all. Even so, we find that poor alignment, whilst degrading the measurement, does introduce a systematic error into the CTR. In the same figure we see the CTR improves (as in decreases) from unwrapped, to partially wrapped and finally to wrapped. This is due to increased light output as seen in figure \ref{fig:lightoutput-20}. In [figure 8]\cite{r_Paganoni_Pauwels_et_al__2011} a CTR of 210ps is seen for the $2\times2\times20$mm$^3$ LSO:CeCa scintillator crystal. Again we see that the CTR with DOI information in the wrapped configuration is improved; in this case by 10ps to a mean value of 198\pm4$ps.