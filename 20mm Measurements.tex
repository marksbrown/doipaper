\subsection{20mm LSO:CeCa Measurements}
To determine to what degree the material and scintillator crystal length contributes to the timing and energy performance we repeat the above measurements with a polished $2\times2\times20$mm$^3$ Ca-co-doped LSO:Ce scintillator crystal. In addition we consider a third `partially wrapped' configuration; namely that we wrap the side faces but leave the face opposing the photodetector unwrapped. In doing so we will reduce the light output and thus the contribution from the backwards reflecting mode. We collect each measurement for 90 minutes.

In figure \ref{fig:confinement-20} the confinement is consistent across all DOI except for the unwrapped configuration. In this case we see a gradual drop off in the number of $\gamma\gamma$ events recorded with increasing DOI. This is due to poor vertical alignment. Due to this, the confinement region lies to the side of the scintillator crystal at higher DOI not within it. In figure \ref{fig:ctrdoi-20} we see that this results in an increasingly larger error in the CTR until not enough events are collected to accurately determine the value at all. Even so, we find that poor alignment whilst degrading the error in the measurement, does not introduce a systematic shift into the CTR. Thus as long as the number of $\gamma\gamma$ events collected is high then alignment is not a critical parameter. 

In figure \ref{fig:ctrdoi-20} we see the CTR improves (as in decreases) with increasing amounts of wrapping; from unwrapped, to partially wrapped and finally wrapped. This is due to increased light output as seen in figure \ref{fig:lightoutput-20}. In [figure 8] \cite{r_Paganoni_Pauwels_et_al__2011} a CTR of 210ps is seen for the $2\times2\times20$mm$^3$ LSO:CeCa scintillator crystal. In the wrapped DOI measurement we see a value of 198.3$\pm$4.4ps. In the previous measurements we see values of 225.0$\pm$3.1ps for the standard coincidence to 239$\pm$4ps in the DOI coincidence setup. Both measurements indicate no improvement to the CTR, within error, is expected for inclusion of confinement information.