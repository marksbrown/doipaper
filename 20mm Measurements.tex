\subsection{20 mm LSO:CeCa Measurements}
\label{sec:20mm}

To determine to what degree the material and scintillator crystal length contributes to the timing and energy performance we repeat the 30mm measurements with a $2\times2\times20$ mm$^3$ Agile Ca-co-doped LSO:Ce scintillator crystal. Additionally we consider a third `partially wrapped' configuration; namely that we wrap the side faces but leave the face opposing the photodetector unwrapped. In doing so we expect to reduce the light output and thus the contribution from the backwards reflecting mode. Each measurement is collected for 90 minutes with an additional 5 minutes ignored at the beginning to ensure temperature stability. In doing so we wish to determine if the null relationship between the CTR and DOI is consistent across more potential variables  to determine any weakness, if any, in our conclusions thus far.

\subsubsection{CTR Results}
In figure \ref{fig:confinement-20} the number of detected $\gamma\gamma$ pairs is constant with DOI except for the unwrapped configuration. In this case we see a gradual drop off in the number of $\gamma\gamma$ events recorded with increasing DOI due to poor vertical alignment of the scintillator crystal. As the DOI is increased, the confinement region will drift outside the scintillator crystal and thus lead to a reduced number of $\gamma\gamma$ events detected in the 90 minutes per measurement. In figure \ref{fig:ctrdoi-20} we see that this results in an increasingly larger error in the CTR until not enough events are collected to accurately determine the value at all. Even so, we find that poor alignment, whilst degrading the error in the measurement, does not introduce a systematic shift into the CTR. Thus as long as the number of $\gamma\gamma$ events collected is high then alignment is not a critical parameter. 
