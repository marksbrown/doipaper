\section{Depth Of Interaction Coincidence}
The standard coincidence apparatus, as shown in figures \ref{fig:doi-ctr} and \ref{fig:actualsetup}, is altered in two key respects . Firstly the right photodetector is placed within a 3D-printed clamp designed to hold the scintillator crystal vertically with respect to the reference detector. Secondly the Na22 source is placed much closer to the vertically aligned scintillator detector than the reference. As in the standard apparatus both scintillator crystals are coupled to their respective photodectors, Hamamatsu MPPC S10931-050P, using Rhorosil 47A optical grease. 

The size of the DOI region is determined by the separation distances between scintillator detectors and the Na22 source.  The source is placed 5mm from the scintillator crystal under investigation. The scintillator detector under investigation is placed 40mm from the source, unless otherwise stated. The source has a diameter of 1mm. For a source much closer to the scintillator detector under interest than to the reference detector, the DOI region will tend to the width of the source. As shown in figures \ref{fig:confinement} and \ref{fig:confinement-20}, this is a valid assumption for our apparatus on the provision the scintillator crystal is aligned vertically in the axis of the apparatus.

In this section measurements are presented for two identical polished Proteus LYSO:Ce $2\times2\times30$mm$^3$ scintillator crystals. Measurements are performed at 2.5mm increments alternating between the two scintillator crystals. This is to determine the contribution, if any, of systematic errors introduced by differences in coupling, alignment and surface finish of the crystal within the system. Each measurement is repeated with and without PTFE (Teflon) tape. The Teflon tape covers all faces, bar that in contact with the photodetector, with several tight bound layers to increase adhesion. The two measurements will be referred to as the wrapped and unwrapped configuration respectively.
