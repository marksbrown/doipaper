\section{Depth Of Interaction Coincidence}
\subsection{Overview}
The standard coincidence apparatus, as shown in figure \ref{fig:doi-ctr}(a) is altered in two key respects. Firstly the right photodetector is placed within a 3D-printed clamp designed to hold the scintillator crystal which is held vertically with respect to the reference detector. Secondly the $^{22}$Na source is placed much closer to the vertically aligned scintillator detector than the reference. As in the standard apparatus both scintillator crystals are coupled to their respective photodetectors, Hamamatsu MPPC S10931-050P SiPMs, using Rhodorsil 47V optical grease. 

The size of the confinement region is primarily determined by the separation distances between scintillator detectors and the $^{22}$Na source. The source is placed 5mm from the scintillator crystal under investigation. The reference scintillator detector is a further 40mm on the opposite side from the source, unless otherwise stated. As the $^{22}$Na cylinder is not a point source, its finite size of 1 mm$^3$ gives a minimum to the confinement region. For a source much closer to the scintillator detector under interest than to the reference detector, the confinement region will tend to the width of the source. 

To determine the size of the confinement region we can exploit the fact that the scintillator detector will detect a fixed number of events per unit time if the volume of scintillator crystal does not change. Therefore for the same measurement and same confinement region we can assume a uniform number of events, regardless of DOI. Furthermore if the confinement region passes outside the scintillator crystal, the number of $\gamma\gamma$ events will drop until electronic collimation prevents any correlations from being detected. In this we assume good alignment of the scintillator crystal with respect to the central axis of the coincidence apparatus. We represent this described behaviour as a convolution between a uniform distribution and a Gaussian distribution. The uniform distribution has a width corresponding to the scintillator crystal length and an amplitude corresponding to the mean number of detected $\gamma\gamma$ events. The FWHM of the normal distribution corresponds to the confinement region; In this case taken as 1mm. As shown in figures \ref{fig:confinement} and \ref{fig:confinement-20} as a black-dotted line this is a valid assumption for our apparatus on the provision the scintillator crystal is properly aligned.

