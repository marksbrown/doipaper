\section{Depth Of Interaction Coincidence}
\subsection{Overview}
The standard coincidence apparatus, as shown in figures \ref{fig:doi-ctr} and \ref{fig:actualsetup}, is altered in two key respects . Firstly the right photodetector is placed within a 3D-printed clamp designed to hold the scintillator crystal vertically with respect to the reference detector. Secondly the Na22 source is placed much closer to the vertically aligned scintillator detector than the reference. As in the standard apparatus both scintillator crystals are coupled to their respective photodectors, Hamamatsu MPPC S10931-050P, using Rhorosil 47A optical grease. 

The size of the DOI region is determined by the separation distances between scintillator detectors and the Na22 source.  The source is placed 5mm from the scintillator crystal under investigation. The scintillator detector under investigation is placed 40mm from the source, unless otherwise stated. The source has a diameter of 1mm. For a source much closer to the scintillator detector under interest than to the reference detector, the DOI region will tend to the width of the source. 

For good alignment of the scintillator crystal with respect to the central axis of the apparatus, the total number of counts per measurement should be constant as the solid angle coverage of the DOI region as it moves is the same. The only exception to this will be near the ends of the scintillator crystal where the DOI region will edge outside the scintillator crystal and thus lead to a reduced region size and thus fewer measured events.  We represent this behaviour as a convolution between a uniform distribution, with amplitude corresponding to the mean number of detected $\gamma\gamma$ events, and a normal distribution with standard deviation of 1mm. As shown in figures \ref{fig:confinement} and \ref{fig:confinement-20} as a black-dotted line this is a valid assumption for our apparatus on the provision the scintillator crystal is aligned vertically in the axis of the apparatus.

