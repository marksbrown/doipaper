\section{Depth Of Interaction Coincidence}
The apparatus as described in \cite{arron_Meyer_Pauwels_Lecoq_2012} is altered in several respects as shown in \ref{fig:actualsetup}. Firstly the right photodetector is placed within a 3D-printed clamp, shown in figure \ref{fig:actualsetup}, designed to hold the scintillator crystal vertically with respect to the reference detector. Secondly the Na22 source is moved to a 5mm separation distance from the scintillator crystal under investigation. As in the standard apparatus both scintillator crystals are coupled to the Hamamatsu MPPC S10931-050P SiPM photodetectors using Rhorosil 47A optical grease. 

The standard timing coincidence apparatus is altered in two key ways. Firstly the scintillator detector under interest is arranged perpendicular to the reference detector. Secondly the gamma ray source is positioned near to the scintillator detector. This arrangement is shown schematically in figure \ref{fig:doi-ctr}. Electronic collimation due to $\gamma\gamma$ correlation defined a DOI region within the scintillator crystal under investigation. By moving the scintillator detector vertically, with respect to the reference detector and source, the DOI region will move. The size of the DOI region is determined by the separation distances between scintillator detectors and the source. In this work the Na22 source has a diameter of 1mm. Where the source is much closer to the scintillator detector under interest than to the reference detector, the DOI region will tend to the width of the source.

In this section measurements are presented for two identical polished Proteus LYSO:Ce $2\times2\times30$mm$^3$ scintillator crystals. Measurements are performed at 2.5mm increments alternating between the two scintillator crystals. This is to determine the contribution, if any, of systematic errors introduced by differences in coupling, alignment and surface finish of the crystal within the system. Each measurement is repeated with and without PTFE (Teflon) tape. The Teflon tape covers all faces, bar that in contact with the photodetector, with several tight bound layers to increase adhesion. The two measurements will be referred to as the wrapped and unwrapped configuration respectively.
