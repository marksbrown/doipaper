\section{Method}
The standard coincidence apparatus is altered in two key ways. Firstly the scintillator detector under interest, which we will refer to as the DOI scintillator detector, is arranged perpendicular to the reference detector. Secondly the gamma ray source is moved closer to the right detector. This arrangement is shown schematically in figure \ref{fig:doi-ctr}. Due to the method of generation of the $\gamma\gamma$ pairs, the DOI region is confined by electronic collimation alone. That is to say, only gamma ray photons interacting within the chosen DOI region will correlate in time and thus  contribute to the time resolution. The size of the DOI region is determined by the seperation distances $x_1$ and $x_2$ as well as the size of the source. In this work the source has a diameter of 1mm. For cases where $\frac{x_1}{x_2}>>1$ the DOI region will tend to a width of that of the source; in this case 1mm.

By moving the right detector vertically, with respect to the left detector and source, the DOI region will move through the scintillator crystal. By doing so we can determine the relationship between the time resolution, the DOI and the scintillator crystal length.