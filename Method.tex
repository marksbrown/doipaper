\section{Method}
\subsection{Overview}
The coincidence apparatus used in this paper is comprised of two Hamamatsu MPPC S10931-050P SiPM connected to CERN-developed NINO discriminators from which the energy and timing information of individual pulses are collected using a LeCroy DDA 735Zi high-bandwidth oscilloscope. Scintillator crystals are coupled to the SiPM photodetectors using Rhorosil 47A optical grease to improve coupling. Prior to coupling all scintillator crystals are cleaned using methanol. Wrapped scintillator crystals are tightly bound in many layers of Teflon to ensure good coupling between the scintillator crystal and wrap. All scintillator crystals are handled with carbon-tipped tweezers to prevent formation of surface defects which may degrade the scintillator crystal performance. The coincidence apparatus is held within a temperature stable chamber to maintain stability of photodetector performance. Further to this, the first 5 minutes of each measurement is discarded due to any potential contribution of temperature variation. The optimal threshold and bias values of the SiPMs were determined by parameter sweep and are given in table \ref{tab:optimumparam}. An excellent description of the experimental method can be found in \cite{ch_Meyer_Pizzichemi_Lecoq_2013}.

