\section{Method}
\subsection{Overview}
The coincidence apparatus used in this paper is comprised of two Hamamatsu MPPC S10931-050P SiPM connected to CERN-developed NINO discriminators from which the energy and timing information of individual pulses are collected using a LeCroy DDA 735Zi high-bandwidth oscilloscope. Scintillator crystals are coupled to the SiPM photodetectors using Rhorosil 47A optical grease to improve coupling. Prior to coupling all scintillator crystals are cleaned using methanol. Wrapped scintillator crystals are tightly bound in many layers of Teflon to ensure good coupling between the scintillator crystal and wrap. All scintillator crystals are handled with carbon-tipped tweezers to prevent formation of surface defects which may degrade the scintillator crystal performance. The coincidence apparatus is held within a temperature stable chamber to maintain stability of photodetector performance. Further to this, the first 5 minutes of each measurement is discarded due to any potential contribution of temperature variation. The optimal threshold and bias values of the SiPMs were determined by parameter sweep and are given in table \ref{tab:optimumparam}.

\subsection{Processing Data}
There are two principal sources of systematic error we wish to correct for in $\gamma\gamma$ data collected from timing coincidence measurements. Firstly low-energy ($\lesssim$1MeV) gamma ray photons may interact with matter in two ways. We wish to select events which interact solely by the photoelectric effect to ensure there is no additional time ambiguity introduced by scattering within the system. To do this we fit to the photopeak generated by the total absorption of the gamma ray photon within the scintillator detector and exclude events outside a $2\sigma$ window about the peak location. The narrow range is chosen to drastically reduce the contribution of overlapping Compton interactions despite losing some photoelectric events. Secondly, the finite bandwidth of the 40GS/s high-bandwidth oscilloscope will lead to multiple `edges' recorded per sampling period. This is due to overlapping independent interactions within the scintillator detector. As a given logical pulse will generate two such edges, an ambiguity in the arrival time and energy of independent is introduced. Thus by selecting for sampling periods containing only two edges we remove this error. 

Events matching this criteria are histogrammed to produced figure XX

$$
\text{CTR} = 4\sqrt{\ln{2}}\sqrt{\sigma-\sigma_\textrm{ref}^2}
$$

where $\sigma_\text{ref}$ is the reference time resolution subtracted in quadrature. A value of $43\pm4$ps was measured using the standard coincidence apparatus as given in \cite{ch_Meyer_Pizzichemi_Lecoq_2013}



