\section{Method}
The standard timing coincidence apparatus is altered in two key ways. Firstly the scintillator detector under interest is arranged perpendicular to the reference detector. Secondly the gamma ray source is positioned near to the scintillator detector. This arrangement is shown schematically in figure \ref{fig:doi-ctr}. Electronic collimation due to $\gamma\gamma$ correlation defined a DOI region within the scintillator crystal under investigation. By moving the scintillator detector vertically, with respect to the reference detector and source, the DOI region will move. The size of the DOI region is determined by the separation distances between scintillator detectors and the source. In this work the Na22 source has a diameter of 1mm. Where the source is much closer to the scintillator detector under interest than to the reference detector, the DOI region will tend to the width of the source.

Two Hamamatsu MPPC S10931-050P SiPM are coupled to CERN-developed NINO discriminators. From these the energy and timing information of individual pulses are collected using a LeCroy DDA 735Zi high-bandwidth oscilloscope. A full description of the electronics used can be found in \cite{arron_Meyer_Pauwels_Lecoq_2012}.