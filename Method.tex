\section{Method}
\label{sec:method}
\subsection{Overview}
The timing coincidence apparatus used in this paper is comprised of two Hamamatsu MPPC S10931-050P SiPMs connected to CERN-developed NINO leading-edge discriminators via analogue amplifiers. From which the energy and timing information of individual pulses are collected using a LeCroy DDA 735Zi high-bandwidth oscilloscope. The coincidence apparatus is held within a temperature-controlled chamber to maintain stability of photodetector performance. Further to this, the first 5 minutes of each measurement is discarded due to any potential contribution of temperature variation.

Scintillator crystals are coupled to the SiPM photodetectors using Rhodorsil 47V optical grease to improve light output. The refractive indices of L(Y)SO and the optical grease are approximately 1.8 \cite{Erdei2012} and 1.4 \cite{rhodorsilgrease} respectively. Wrapped scintillator crystals are tightly bound in many layers of Teflon to ensure good coupling between the scintillator crystal and wrap. Prior to wrapping and usage, all scintillator crystals are cleaned using isopropyl alcohol. All scintillator crystals are handled with carbon-tipped tweezers to prevent formation of surface defects which may degrade the scintillator crystal performance. 

The optimal threshold and bias values of the SiPMs were determined by parameter sweep and are given in table \ref{tab:optimumparam}. An excellent description of the experimental method can be found in \cite{ch_Meyer_Pizzichemi_Lecoq_2013}.
