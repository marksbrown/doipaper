In figures \ref{fig:energyresolution} and \ref{fig:lightoutput} we can see the energy resolution and right photopeak centroid of the scintillator detector with DOI. The latter corresponds to the absolute light output arriving at the photodetector. Firstly we see the light output and energy resolution are better in the wrapped configuration compared to the unwrapped for both scintillator crystals as expected. For a DOI greater than 5mm, the mean energy resolutions are 16.52$\pm$0.02 and 13.92$\pm$0.01 for the unwrapped and wrapped configurations respectively. Secondly we notice that the wrapped measurements show a systematic variation in the photopeak centroid, most likely due to differences in wrapping or coupling. Interestingly this pattern is also observed in figure \ref{fig:ctrvsdoi} showing a poorer CTR for the `30B Wrapped' compared to the `30A Wrapped'. Thirdly we notice that the light output is about 20\% higher in the wrapped configuration with a minor drop off in both configurations with increasing DOI. We attribute this to longer path lengths through the scintillator crystal at higher DOI and therefore a greater chance of escape via Lobe reflection\cite{Janecek_Moses_2010} or losses via absorption.

In all measurements the energy resolution is seen to be at it's poorest for low DOI values, despite a weak variation seen in the light output. This indicates a broadening of the photopeak at low DOI values as no large change in the absolute light output is observed. Given that this broadening occurs only at low DOI this implies the cause is due to the geometry of the scintillator crystal. Specifically at low DOI, the solid angle of generated light reaching the photodetector without interacting with the side faces is high. As the photopeak is not seen to shift, thus the same light output in each measurement, we can see no saturation effects due to the SiPM. Therefore
the energy resolution degradation at low DOI can be attributed solely to the confinement of gamma ray photon interactions near to the photodetector.