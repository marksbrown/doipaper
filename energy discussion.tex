In figures \ref{fig:energyresolution} and \ref{fig:lightoutput} we can see the energy performance of the scintillator detector with DOI. We define the energy resolution as
\begin{align}
\sigma_E = \frac{\Delta E}{E}
\end{align}
where $\Delta E$ is the FWHM of the photopeak with the photopeak centroid $E$. The energy resolution is typically given as percentage. We first notice that the wrapped measurements show a systematic variation, most likely due to differences in wrapping or coupling. Secondly we see the light output is about 20\% higher in the wrapped configuration with a minor drop off in both configurations with DOI. We attribute this to losses of light at surface interactions. In all measurements the energy resolution is seen to be at it's poorest for low DOI values, despite no such variation seen in the light output. This indicates a broadening of the photopeak at low DOI values which can be attributed to 