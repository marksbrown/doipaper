\subsection{Processing Data}
There are two principal sources of systematic error we wish to correct for in $\gamma\gamma$ data collected from timing coincidence measurements. Firstly low-energy ($\lesssim$1MeV) gamma ray photons may interact with matter in two ways. We wish to select events which interact solely by the photoelectric effect to ensure there is no additional time ambiguity introduced by scattering within the system. To do this we fit to the photopeak generated by the total absorption of the gamma ray photon within the scintillator detector and exclude events outside a $2\sigma$ window about the peak location. The narrow range is chosen to drastically reduce the contribution of overlapping Compton interactions despite losing some photoelectric events. Secondly, the finite bandwidth of the 40GS/s high-bandwidth oscilloscope will lead to multiple `edges' recorded per sampling period. This is due to overlapping independent interactions within the scintillator detector. As a given logical pulse will generate two such edges, an ambiguity in the arrival time and energy of independent is introduced. Thus by selecting for sampling periods containing only two edges we remove this error. Events matching these criteria are grouped to produce a subset of data solely due $\gamma\gamma$ correlations. The delay distribution can then be drawn from this data; for example figure \ref{fig:delaydist}. For two identical photodetectors the FWHM of the Gaussian fit is the coincidence time resolution, or CTR. Such that

\begin{align}
\text{CTR} = 2\sqrt{\ln{2}}\sigma
\end{align}
where $\sigma$ is the time resolution of a single scintillator detector. In cases where we use a known reference scintillator detector, the CTR is determined by subtraction in quadrature and a subsequent scaling such that
\begin{align}
\text{CTR} = 4\sqrt{\ln{2}}\sqrt{\sigma-\sigma_\textrm{ref}^2}
\label{eqn:ctrfromref}
\end{align}
where $\sigma_\text{ref}$ is the reference time resolution.