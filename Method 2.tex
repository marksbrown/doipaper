\subsection{Processing Data}
There are two principal sources of systematic error to correct for in $\gamma\gamma$ data collected from timing coincidence measurements. Firstly low-energy ($\lesssim$1MeV) gamma ray photons may interact with matter in two ways. We wish to select events which interact solely by the photoelectric effect to ensure collected events are unambiguously correlated in time. This property, namely electronic collimation, allows to use to implicitly know true events were excited from within the confinement region. This is shown in figure \ref{fig:doi-ctr}.

$\gamma$ events falling within $2\sigma$ of the photopeak centroid of their respective energy spectra are selected. This narrow range is chosen to drastically reduce the contribution of overlapping Compton interactions despite losing some photoelectric events. Events matching these criteria are grouped to produce a subset of data solely due $\gamma\gamma$ correlations. The difference in arrival time between $\gamma\gamma$ pairs is histogrammed to produce a Gaussian distribution. This will be referred to as the (relative) delay peak. For two identical photodetectors the FWHM of the delay peak is defined as the coincidence time resolution (CTR), such that
%Secondly, the finite bandwidth of the 40GS/s high-bandwidth oscilloscope will lead to multiple events recorded per sampling period. This is due to overlapping independent interactions within the scintillator detector. To reduce noise (and ambiguity) we select for periods containing two $\gamma$ detections only. Events matching these criteria are grouped to produce a subset of data solely due $\gamma\gamma$ correlations. ----First sentence is wrong. The paragraph seems OK to me without these sentences.

\begin{align}
\text{CTR} = 4\sqrt{\ln{2}}\sigma 
\end{align}

where $\sigma$ is the time resolution of a single scintillator detector. This relationship is due to the convolution of two identical Gaussian distributions corresponding to the individual scintillator detectors. In cases where we use a reference scintillator detector with a known time resolution, the CTR of an unknown scintillator detector is determined by subtraction in quadrature and a subsequent scaling such that

\begin{align}
\text{CTR} = 2\sqrt{\ln{2}}\sqrt{\sigma^2-\sigma_\textrm{ref}^2}
\end{align}

where $\sigma_\text{ref}$ is the reference time resolution. All CTR values in this paper are given in picoseconds.

\subsection{Analysis of Data}
The parameters describing the location and scale of the Gaussian distributions, namely the photopeaks and delay peak per measurement, were found by weighted least-squared fit. The error per bin was assumed Poissonian and thus taken as the square root of the number of measurements per bin. The standard error in the fit parameters were determined by the bootstrap\cite{degroot2012probability}. The full code used to perform the peak detection, peak fitting, parameter error determination and image \& table generation can be found at \href{https://github.com/marksbrown/ProcessingCTRData}{https://github.com/marksbrown/ProcessingCTRData}. An online version of this paper can be found at \cite{Brown2014}.