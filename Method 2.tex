\subsection{Standard Coincidence Measurements}
In the standard coincidence apparatus, the two scintillator detectors are placed opposite one another with the source an equal distance between the two. In the first set of results, given in section \ref{sec:standardctr}, we determine the CTR of Proteus LYSO:Ce scintillator crystals of lengths 5, 10, 15, 20 and 30mm. All crystals have the same cross section of $2\times2$mm$^2$.



\subsection{Depth Of Interaction Coincidence Measurements}
The apparatus as described in \cite{arron_Meyer_Pauwels_Lecoq_2012} is altered in several respects as shown in \ref{fig:actualsetup}. Firstly the right photodetector is placed within a 3D-printed clamp, shown in figure \ref{fig:actualsetup}, designed to hold the scintillator crystal vertically with respect to the reference detector. Secondly the Na22 source is moved to a 5mm separation distance from the scintillator crystal under investigation. As in the standard apparatus both scintillator crystals are coupled to the Hamamatsu MPPC S10931-050P SiPM photodetectors using Rhorosil 47A optical grease. 

The standard timing coincidence apparatus is altered in two key ways. Firstly the scintillator detector under interest is arranged perpendicular to the reference detector. Secondly the gamma ray source is positioned near to the scintillator detector. This arrangement is shown schematically in figure \ref{fig:doi-ctr}. Electronic collimation due to $\gamma\gamma$ correlation defined a DOI region within the scintillator crystal under investigation. By moving the scintillator detector vertically, with respect to the reference detector and source, the DOI region will move. The size of the DOI region is determined by the separation distances between scintillator detectors and the source. In this work the Na22 source has a diameter of 1mm. Where the source is much closer to the scintillator detector under interest than to the reference detector, the DOI region will tend to the width of the source.


