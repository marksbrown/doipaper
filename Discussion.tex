\section{Discussion}
\label{sec:discussion}
\subsection{Key Results}
Of the results presented in this paper, indicate at most a weak dependence of the CTR upon the DOI. We conclude that the weight of evidence presented in the paper lends itself to the conclusion that no relationship, within experimental error, was found between the coincidence time resolution and depth of interaction. This result is in agreement with \cite{Bircher_Shao_2012} for polished scintillator crystals. The mean CTR recorded using the DOI timing coincidence apparatus was consistently poorer than equivalent measurements performed using the standard apparatus using the same equipment. As this held across (wrapped) geometries and scintillator crystal materials for it indicates that the difference is either due to the bias and threshold voltages chosen or a systematic difference in the performance due to inclusion of DOI information. As the reference scintillator crystal CTR values are shown to be in good agreement with previous measurements in both the standard and DOI apparatus, we conclude the confinement region has a negative effect upon the CTR. We also note that the difference in the CTR between configurations was suppressed in the DOI apparatus.

The results presented in [Figure 8]\cite{Moses_Derenzo_1999} show the delay peak centroid with DOI shifting in the same manner as presented in figures \ref{fig:centroidposition} and \ref{fig:centroidposition-20}. In the paper the gradient is presented as an effective refractive index such that $n=mc$ where $m$ is the gradient of the fitted line (in SI units) and $c$ is the speed of light. For the 30mm measurements, fitting to DOI below 20mm, we find 3.6$\pm$1.5 and 3.4$\pm$1.2 for the unwrapped and wrapped configurations respectively.  For the 20mm measurements, fitting to DOI below 10mm, we find 3.9$\pm$1.0, 5.0$\pm$0.6 and 4.4$\pm$1.1 for the unwrapped, partially wrapped and wrapped configurations respectively. In \cite{Moses_Derenzo_1999} a value of 3.9 is presented for polished scintillator crystals in good agreement with our calculated values for the effective refractive index. Of note is the higher value for the partially wrapped configuration. As a significant portion of photons in the `backward propagating' mode will reflect from the rear at higher angles, as photons at shallower angles now escape, we can conclude that the light is taking longer to reach the photodetector due to this. The curve shape is predominantly attributed to finite propagation time of information through the scintillator crystal. Given that the plateau occurs at high DOI we would conclude that the variation in the time between the forward and backward modes becomes negligible and as such no longer moves the delay peak.

The light output and energy resolution are seen in both the 20mm and 30mm measurements to degrade weakly with increasing DOI. This behaviour is attributed to losses from a higher path length through the scintillator crystal. The severe penalty in the energy resolution at very low DOI is primarily due to the large potential variation in the photon travel time to the photodetector.
