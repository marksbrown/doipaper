\section{Discussion}
Of the results presented in this paper, either a weak dependence of the CTR upon the DOI was seen or none was seen. We conclude that the weight of evidence presented in the paper lends itself to the conclusion that no relationship, within experimental error, was found between the coincidence time resolution and depth of interaction. This result is in agreement with \cite{Bircher_Shao_2012} for polished scintillator crystals. 


Of note are the results presented in [Figure 8]\cite{Moses_Derenzo_1999} showing the delay peak centroid with DOI shifting in the same manner as presented in figures \ref{fig:centroidposition} and \ref{fig:centroidposition-20}. In the paper the gradient is presented as an effective refractive index such that $n=mc$ where $m$ is the gradient of the fitted line (in SI units) and $c$ is the speed of light. For the 30mm measurements, fitting to DOI below 15mm, we find 4.4$\pm$3.4 and $4.3$\pm$2.3 for the unwrapped and wrapped configurations respectively. 


For the 20mm measurements, fitting to DOI below 10mm, we find 3.9$\pm$1.0, 5.0$\pm$0.6 and 4.4$\pm$1.1 for the unwrapped, partially wrapped and wrapped configurations respectively. In \cite{Moses_Derenzo_1999} a value of 3.9 is presented for polished scintillator crystals in good agreement with our values. Of note is the higher value for the partially wrapped configuration. As a significant portion of photons in the `backward propagating' mode will reflect from the rear at higher angles, as photons at shallower angles now escape, we can conclude that the light is taking longer to reach the photodetector due to this. As such light travelling both directions from it's emission site are contributing to the timing performance of the scintillator detector. We conclude that the surface finish is the predominant factor altering the light transport to the photodetector; whereas the wrapping is the predominant factor to increasing the overall light output. In this manner, the `inner' and `outer' surfaces both contribute to the performance of the scintillator detector.