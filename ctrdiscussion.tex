In figure \ref{fig:ctrvsdoi} the CTR (in ps) against DOI (in (mm) per sample and configuration is given. In table \ref{tab:doictrresults} the values given for the timing and energy performance are averaged across the DOI. Firstly we note that no clear relationship between CTR and DOI is visible. The reduced chi-squared fit shows values close to unity for fitting to the weighted mean, indicating no relationship between CTR and DOI in both crystals and configurations. Secondly the CTR measurements from the wrapped configuration are consistently better than those from the unwrapped. The differences being 8$\pm$5ps and 18$\pm$6ps for 30A and 30B respectively. This difference is much smaller than that which we would expect in the standard CTR measurement. It's seen in [Table IV]\cite{r_Paganoni_Pauwels_et_al__2011} that the difference between wrapped and unwrapped CTR is about 33\%. In this case the differences are 3$\pm$2\% and 7$\pm$2\%. This implies that knowledge of the excitation position within the standard coincidence apparatus for an unwrapped scintillator crystal would reduce the measured CTR by at least 25\%. 

