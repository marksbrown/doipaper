In figure \ref{fig:ctrvsdoi} the CTR (in ps) against DOI (in (mm) per sample and configuration is given. In table \ref{tab:doictrresults} the mean values for timing and energy performance are given. It can be seen that the CTR of the wrapped samples is superior to the unwrapped by 20ps. The difference between the two CTR measurements, 8\% averaged between the measurements, is much smaller than that which we would expect in the standard CTR measurement. It's seen in [Table IV]\cite{r_Paganoni_Pauwels_et_al__2011} that the difference between wrapped and unwrapped CTR is about 33\%. This implies that knowledge of the excitation position within the standard coincidence apparatus for an unwrapped scintillator crystal would reduce the CTR by 20\%. 

In table \ref{tab:ctrfitresults} we perform two polynomial fits; namely the zeroth and 1st order. We determine the quality of these fits using $\chi^2$\footnote{I'm tempted to replace these with correlation coefficients} where values close to unity are good fits. $\chi^2_\text{nofit}$ is the reduced chi-squared test for no relationship between the CTR and DOI. $\chi^2_\text{linearfit}$ is the reduced chi-squared test for a linear relationship between the CTR and DOI. In the first instance we see the values are close to unity. Furthermore the error in the intercept$_\text{nofit}$ is larger than the peak to peak (PTP) change in the CTR values, indicating that there is no relationship between the CTR and DOI in all four data sets. Expanding to linear fits the $\chi^2_\text{linearfit}$ values are, on the whole, lower than unity indicating a parameter over specification.