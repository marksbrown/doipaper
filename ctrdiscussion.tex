In figure \ref{fig:ctrvsdoi} the CTR (in ps) with sample, DOI (in (mm) and configuration is given. In table \ref{tab:doictrresults} the mean values for timing and energy performance are given. It can be seen that the CTR of the wrapped samples is superior to the unwrapped by approximately 20ps. The difference between the two CTR measurements, 8\% averaged between the measurements, is much smaller than that which we would expect in the standard CTR measurement. It's seen in [Table IV]\cite{r_Paganoni_Pauwels_et_al__2011} that the difference between wrapped and unwrapped CTR is about 33\%. This implies that knowledge of the DOI can reduce the CTR by 20\% within the standard coincidence apparatus.

In table \ref{tab:ctrfitresults} we perform intercept and linear fits upon each sample and configuration shown in figure \ref{fig:ctrvsdoi}. It can be seen that $\chi^2_\text{nofit}$, the reduced chi-squared test for no relationship between the CTR and DOI is close to unity indicating an acceptable fit. For the reduced chi-squared test for a linear relationship, $\chi^2_\text{linearfit}$, it can be seen that the values are, on the whole, lower than unity. Given the gradient parameters, this would suggest that there is either a weak improvement in the CTR with DOI or no relationship whatsoever. Finally we look at the peak to peak change with DOI. The change is larger in every measurement than the error in the mean CTR value given in table \ref{tab:doictrresults}. Therefore we conclude that there is a weak dependence on the CTR with DOI.