In figure \ref{fig:ctrvsdoi} the CTR (in ps) against DOI (in mm) per sample and configuration is given. In table \ref{tab:doictrresults} the values given for the timing and energy performance are averaged across the DOI. Firstly we note that no clear relationship between CTR and DOI is visible. The reduced chi-squared fit shows values close to unity for fitting to the weighted mean, indicating no relationship between CTR and DOI in both crystals and configurations. Secondly the CTR measurements from the wrapped configuration are consistently better than those from the unwrapped. The differences being 15$\pm$3 ps and 25$\pm$6 ps for 30A and 30B respectively. This difference is much smaller than that which we would expect in the standard CTR measurement. For instance it is seen in [Table IV]\cite{r_Paganoni_Pauwels_et_al__2011} that the difference in the CTR between wrapped and unwrapped configurations is approximately 33\%. The differences for 30A and 30B are 6$\pm$1\% and 10$\pm$2\%. This implies that knowledge of the excitation position within the standard coincidence apparatus for an unwrapped scintillator crystal would reduce the measured CTR by at least 23\%. We would predict this behaviour is due to a reduction in the variance of the photon travel time  to the photodetector across multiple gamma ray photon detections. With DOI information, and limited diffusion in a polished unwrapped scintillator crystal, the photon travel time variance will be low.

