('30B', 'wrapped') 231.7+/-2.9
('30A', 'wrapped') 246.0+/-2.3
('30A', 'unwrapped') 260.7+/-2.2
('30B', 'unwrapped') 257+/-5

In figure \ref{fig:ctrvsdoi} the CTR (in ps) against DOI (in (mm) per sample and configuration is given. 

In table \ref{tab:doictrresults} the values given for the timing and energy performance are averaged across the DOI. Firstly we note that no clear relationship between CTR and DOI is visible. Secondly the CTR measurements from the wrapped configuration are consistently better than those from the unwrapped. The difference being 


It can be seen that the CTR of the wrapped configurations are superior to the unwrapped by 20ps. The difference between the two CTR measurements, 8\% averaged between the measurements, is much smaller than that which we would expect in the standard CTR measurement. It's seen in [Table IV]\cite{r_Paganoni_Pauwels_et_al__2011} that the difference between wrapped and unwrapped CTR is about 33\%. This implies that knowledge of the excitation position within the standard coincidence apparatus for an unwrapped scintillator crystal would reduce the CTR by 20\%. 

