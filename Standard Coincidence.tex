\section{Standard Coincidence}
\label{sec:standardctr}
Prior to the DOI experiment, standard coincidence measurements are made of Proteus LYSO:Ce scintillator crystals with a cross section of $2\times2$mm$^2$ wrapped in PTFE tape for lengths, $L$, of 5, 10, 15, 20 and 30mm. Two identical crystals, which are referred to as $L\text{A}$ and $L\text{B}$, of each length are measured opposite one another. The CTR is then determined as the FWHM of the delay peak directly. In figure \ref{fig:standardctr} we see the expected degradation of the CTR with increasing crystal length \cite{r_Paganoni_Pauwels_et_al__2011}\cite{Wiener_Kaul_Surti_Karp_2010}\cite{Choong_2009}\cite{Gola_Piemonte_Tarolli_2013}\cite{o_Pro_Serra_Tarolli_Zorzi_2011}. The full set of results are given in table \ref{tab:standardctr}. In the table we see that the energy resolution, for both left and right scintillator detectors, is poorer at higher scintillator crystal lengths. This is due to increased variance in the energy recorded and reduced light detected for a 511keV gamma ray photon; likely due to increased path length of photons through the scintillator crystal. Also note that each measurement was conducted for a 15 minutes each. Thus the number of $\gamma\gamma$ events detected increases with the scintillator crystal lengths as expected for increased sensitivity.