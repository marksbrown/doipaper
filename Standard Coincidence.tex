\section{Standard Coincidence}
\label{sec:standardctr}
To begin standard coincidence measurements are made of Proteus LYSO:Ce scintillator crystals with a cross section of $2\times2$mm$^2$ wrapped in PTFE tape for lengths 5, 10, 15, 20 and 30mm. Two crystals of each length are measured opposite one another. The CTR is then determined as the FWHM of the delay distribution directly. In figure \ref{fig:standardctr} we see the expected degradation of the CTR with increasing crystal length \cite{r_Paganoni_Pauwels_et_al__2011}\cite{Wiener_Kaul_Surti_Karp_2010}\cite{Choong_2009}\cite{Gola_Piemonte_Tarolli_2013}\cite{o_Pro_Serra_Tarolli_Zorzi_2011}. The full set of results in given in table \ref{tab:standardctr}. In the results we also see that the energy resolution, for both left and right scintillator detectors, is poorer at higher scintillator crystal lengths. This is due to increased variance in the energy recorded for a 0.511keV gamma ray photon; likely due to increased path length of photons through the scintillator crystal leading to larger variances in the number of photons being recorded per event. Furthermore we see that the total number of $\gamma\gamma$ events recorded increases with scintillator crystal length as we would expect due the increased sensitivity.